\documentclass[]{article}
\usepackage{lmodern}
\usepackage{amssymb,amsmath}
\usepackage{ifxetex,ifluatex}
\usepackage{fixltx2e} % provides \textsubscript
\ifnum 0\ifxetex 1\fi\ifluatex 1\fi=0 % if pdftex
  \usepackage[T1]{fontenc}
  \usepackage[utf8]{inputenc}
\else % if luatex or xelatex
  \ifxetex
    \usepackage{mathspec}
  \else
    \usepackage{fontspec}
  \fi
  \defaultfontfeatures{Ligatures=TeX,Scale=MatchLowercase}
\fi
% use upquote if available, for straight quotes in verbatim environments
\IfFileExists{upquote.sty}{\usepackage{upquote}}{}
% use microtype if available
\IfFileExists{microtype.sty}{%
\usepackage{microtype}
\UseMicrotypeSet[protrusion]{basicmath} % disable protrusion for tt fonts
}{}
\usepackage[margin=1in]{geometry}
\usepackage{hyperref}
\hypersetup{unicode=true,
            pdftitle={Multiple Linear Regresssion},
            pdfauthor={R. Gevorgyan},
            pdfborder={0 0 0},
            breaklinks=true}
\urlstyle{same}  % don't use monospace font for urls
\usepackage{color}
\usepackage{fancyvrb}
\newcommand{\VerbBar}{|}
\newcommand{\VERB}{\Verb[commandchars=\\\{\}]}
\DefineVerbatimEnvironment{Highlighting}{Verbatim}{commandchars=\\\{\}}
% Add ',fontsize=\small' for more characters per line
\usepackage{framed}
\definecolor{shadecolor}{RGB}{248,248,248}
\newenvironment{Shaded}{\begin{snugshade}}{\end{snugshade}}
\newcommand{\KeywordTok}[1]{\textcolor[rgb]{0.13,0.29,0.53}{\textbf{#1}}}
\newcommand{\DataTypeTok}[1]{\textcolor[rgb]{0.13,0.29,0.53}{#1}}
\newcommand{\DecValTok}[1]{\textcolor[rgb]{0.00,0.00,0.81}{#1}}
\newcommand{\BaseNTok}[1]{\textcolor[rgb]{0.00,0.00,0.81}{#1}}
\newcommand{\FloatTok}[1]{\textcolor[rgb]{0.00,0.00,0.81}{#1}}
\newcommand{\ConstantTok}[1]{\textcolor[rgb]{0.00,0.00,0.00}{#1}}
\newcommand{\CharTok}[1]{\textcolor[rgb]{0.31,0.60,0.02}{#1}}
\newcommand{\SpecialCharTok}[1]{\textcolor[rgb]{0.00,0.00,0.00}{#1}}
\newcommand{\StringTok}[1]{\textcolor[rgb]{0.31,0.60,0.02}{#1}}
\newcommand{\VerbatimStringTok}[1]{\textcolor[rgb]{0.31,0.60,0.02}{#1}}
\newcommand{\SpecialStringTok}[1]{\textcolor[rgb]{0.31,0.60,0.02}{#1}}
\newcommand{\ImportTok}[1]{#1}
\newcommand{\CommentTok}[1]{\textcolor[rgb]{0.56,0.35,0.01}{\textit{#1}}}
\newcommand{\DocumentationTok}[1]{\textcolor[rgb]{0.56,0.35,0.01}{\textbf{\textit{#1}}}}
\newcommand{\AnnotationTok}[1]{\textcolor[rgb]{0.56,0.35,0.01}{\textbf{\textit{#1}}}}
\newcommand{\CommentVarTok}[1]{\textcolor[rgb]{0.56,0.35,0.01}{\textbf{\textit{#1}}}}
\newcommand{\OtherTok}[1]{\textcolor[rgb]{0.56,0.35,0.01}{#1}}
\newcommand{\FunctionTok}[1]{\textcolor[rgb]{0.00,0.00,0.00}{#1}}
\newcommand{\VariableTok}[1]{\textcolor[rgb]{0.00,0.00,0.00}{#1}}
\newcommand{\ControlFlowTok}[1]{\textcolor[rgb]{0.13,0.29,0.53}{\textbf{#1}}}
\newcommand{\OperatorTok}[1]{\textcolor[rgb]{0.81,0.36,0.00}{\textbf{#1}}}
\newcommand{\BuiltInTok}[1]{#1}
\newcommand{\ExtensionTok}[1]{#1}
\newcommand{\PreprocessorTok}[1]{\textcolor[rgb]{0.56,0.35,0.01}{\textit{#1}}}
\newcommand{\AttributeTok}[1]{\textcolor[rgb]{0.77,0.63,0.00}{#1}}
\newcommand{\RegionMarkerTok}[1]{#1}
\newcommand{\InformationTok}[1]{\textcolor[rgb]{0.56,0.35,0.01}{\textbf{\textit{#1}}}}
\newcommand{\WarningTok}[1]{\textcolor[rgb]{0.56,0.35,0.01}{\textbf{\textit{#1}}}}
\newcommand{\AlertTok}[1]{\textcolor[rgb]{0.94,0.16,0.16}{#1}}
\newcommand{\ErrorTok}[1]{\textcolor[rgb]{0.64,0.00,0.00}{\textbf{#1}}}
\newcommand{\NormalTok}[1]{#1}
\usepackage{graphicx,grffile}
\makeatletter
\def\maxwidth{\ifdim\Gin@nat@width>\linewidth\linewidth\else\Gin@nat@width\fi}
\def\maxheight{\ifdim\Gin@nat@height>\textheight\textheight\else\Gin@nat@height\fi}
\makeatother
% Scale images if necessary, so that they will not overflow the page
% margins by default, and it is still possible to overwrite the defaults
% using explicit options in \includegraphics[width, height, ...]{}
\setkeys{Gin}{width=\maxwidth,height=\maxheight,keepaspectratio}
\IfFileExists{parskip.sty}{%
\usepackage{parskip}
}{% else
\setlength{\parindent}{0pt}
\setlength{\parskip}{6pt plus 2pt minus 1pt}
}
\setlength{\emergencystretch}{3em}  % prevent overfull lines
\providecommand{\tightlist}{%
  \setlength{\itemsep}{0pt}\setlength{\parskip}{0pt}}
\setcounter{secnumdepth}{0}
% Redefines (sub)paragraphs to behave more like sections
\ifx\paragraph\undefined\else
\let\oldparagraph\paragraph
\renewcommand{\paragraph}[1]{\oldparagraph{#1}\mbox{}}
\fi
\ifx\subparagraph\undefined\else
\let\oldsubparagraph\subparagraph
\renewcommand{\subparagraph}[1]{\oldsubparagraph{#1}\mbox{}}
\fi

%%% Use protect on footnotes to avoid problems with footnotes in titles
\let\rmarkdownfootnote\footnote%
\def\footnote{\protect\rmarkdownfootnote}

%%% Change title format to be more compact
\usepackage{titling}

% Create subtitle command for use in maketitle
\newcommand{\subtitle}[1]{
  \posttitle{
    \begin{center}\large#1\end{center}
    }
}

\setlength{\droptitle}{-2em}

  \title{Multiple Linear Regresssion}
    \pretitle{\vspace{\droptitle}\centering\huge}
  \posttitle{\par}
    \author{R. Gevorgyan}
    \preauthor{\centering\large\emph}
  \postauthor{\par}
      \predate{\centering\large\emph}
  \postdate{\par}
    \date{October 21, 2018}


\begin{document}
\maketitle

\subsection{Data}\label{data}

Here, we employ the CPS1988 data frame collected in the March 1988
Current Population Survey (CPS) by the US Census Bureau and analyzed by
Bierens and Ginther (2001).

These are cross-section data on males aged 18 to 70 with positive annual
income greater than US\$ 50 in 1992 who are not self-employed or working
without pay.

\begin{Shaded}
\begin{Highlighting}[]
\KeywordTok{library}\NormalTok{(AER)}
\end{Highlighting}
\end{Shaded}

\begin{Shaded}
\begin{Highlighting}[]
\KeywordTok{data}\NormalTok{(}\StringTok{"CPS1988"}\NormalTok{)}
\KeywordTok{summary}\NormalTok{(CPS1988)}
\end{Highlighting}
\end{Shaded}

\begin{verbatim}
##       wage            education       experience   ethnicity   
##  Min.   :   50.05   Min.   : 0.00   Min.   :-4.0   cauc:25923  
##  1st Qu.:  308.64   1st Qu.:12.00   1st Qu.: 8.0   afam: 2232  
##  Median :  522.32   Median :12.00   Median :16.0               
##  Mean   :  603.73   Mean   :13.07   Mean   :18.2               
##  3rd Qu.:  783.48   3rd Qu.:15.00   3rd Qu.:27.0               
##  Max.   :18777.20   Max.   :18.00   Max.   :63.0               
##   smsa             region     parttime   
##  no : 7223   northeast:6441   no :25631  
##  yes:20932   midwest  :6863   yes: 2524  
##              south    :8760              
##              west     :6091              
##                                          
## 
\end{verbatim}

\begin{verbatim}
* wage - the wage in dollars per week, 
* education and experience - measured in years
* ethnicity is a factor with levels Caucasian ("cauc") and African-American ("afam"). 
* smsa - indicating residence in a standard metropolitan statistical area (SMSA) 
* region - the region within the United States of America, and
* parttime - whether the individual works part-time.
\end{verbatim}

Note that the CPS does not provide actual work experience. It is
therefore customary to compute experience as \textbf{\emph{age -
education - 6}}; this may be considered potential experience. This
quantity may become negative.

The model of interest is

\textbf{\emph{log(wage) = \(\beta\)\textsubscript{1} +
\(\beta\)\textsubscript{2} experience + \(\beta\)\textsubscript{3}
experience2 +\(\beta\)\textsubscript{4} education
+\(\beta\)\textsubscript{5} ethnicity +\(\varepsilon\) }}

This is a semilogarithmic model, which can be fitted in R using

\begin{Shaded}
\begin{Highlighting}[]
\NormalTok{cps_lm <-}\StringTok{ }\KeywordTok{lm}\NormalTok{(}\KeywordTok{log}\NormalTok{(wage) }\OperatorTok{~}\StringTok{ }\NormalTok{experience }\OperatorTok{+}\StringTok{ }\KeywordTok{I}\NormalTok{(experience}\OperatorTok{^}\DecValTok{2}\NormalTok{) }\OperatorTok{+}\NormalTok{education }\OperatorTok{+}\StringTok{ }\NormalTok{ethnicity, }\DataTypeTok{data =}\NormalTok{ CPS1988)}
\KeywordTok{summary}\NormalTok{(cps_lm)}
\end{Highlighting}
\end{Shaded}

\begin{verbatim}
## 
## Call:
## lm(formula = log(wage) ~ experience + I(experience^2) + education + 
##     ethnicity, data = CPS1988)
## 
## Residuals:
##     Min      1Q  Median      3Q     Max 
## -2.9428 -0.3162  0.0580  0.3756  4.3830 
## 
## Coefficients:
##                   Estimate Std. Error t value Pr(>|t|)    
## (Intercept)      4.321e+00  1.917e-02  225.38   <2e-16 ***
## experience       7.747e-02  8.800e-04   88.03   <2e-16 ***
## I(experience^2) -1.316e-03  1.899e-05  -69.31   <2e-16 ***
## education        8.567e-02  1.272e-03   67.34   <2e-16 ***
## ethnicityafam   -2.434e-01  1.292e-02  -18.84   <2e-16 ***
## ---
## Signif. codes:  0 '***' 0.001 '**' 0.01 '*' 0.05 '.' 0.1 ' ' 1
## 
## Residual standard error: 0.5839 on 28150 degrees of freedom
## Multiple R-squared:  0.3347, Adjusted R-squared:  0.3346 
## F-statistic:  3541 on 4 and 28150 DF,  p-value: < 2.2e-16
\end{verbatim}

The summary reveals that all coefficients have the expected sign, and
the corresponding variables are highly significant (not surprising in a
sample as large as the present one). Specifically, according to this
specification, the return on education is 8.57\% per year.

To illustrate the general procedure for model comparisons, we explicitly
fit the model without ethnicity and then compare both models using
anova()

\begin{Shaded}
\begin{Highlighting}[]
\NormalTok{cps_noeth <-}\StringTok{ }\KeywordTok{lm}\NormalTok{(}\KeywordTok{log}\NormalTok{(wage) }\OperatorTok{~}\StringTok{ }\NormalTok{experience }\OperatorTok{+}\StringTok{ }\KeywordTok{I}\NormalTok{(experience}\OperatorTok{^}\DecValTok{2}\NormalTok{) }\OperatorTok{+}\StringTok{ }\NormalTok{education, }\DataTypeTok{data =}\NormalTok{ CPS1988)}
\KeywordTok{anova}\NormalTok{(cps_noeth, cps_lm)}
\end{Highlighting}
\end{Shaded}

\begin{verbatim}
## Analysis of Variance Table
## 
## Model 1: log(wage) ~ experience + I(experience^2) + education
## Model 2: log(wage) ~ experience + I(experience^2) + education + ethnicity
##   Res.Df    RSS Df Sum of Sq      F    Pr(>F)    
## 1  28151 9719.6                                  
## 2  28150 9598.6  1    121.02 354.91 < 2.2e-16 ***
## ---
## Signif. codes:  0 '***' 0.001 '**' 0.01 '*' 0.05 '.' 0.1 ' ' 1
\end{verbatim}


\end{document}
